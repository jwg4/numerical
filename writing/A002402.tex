\documentclass{article}

\usepackage{hyperref}
\usepackage{amsmath}
\usepackage{amsfonts}
\usepackage{amssymb}

\title{Coefficients for numerical integration as defined by Pickard}
\author{Jack Grahl}

\begin{document}
\maketitle

\section{Pickard's paper}
The article \cite{pickard} provided the sequences A002397 to A002406. Most of these sequences are taken from the tables for $\delta_p(J)$ and $\delta^{\star}_p(J)$. These are defined for all $0 \leq J$ and $0 \leq p \leq J$.

Pickard first defines (in (6a)):
\begin{equation}\label{eqn:beta_integral}
 \beta_j = \frac{1}{j!}\int_0^1 (u + j - 1)^{[j]} du 
\end{equation}
where the integrand is the polynomial defined by:
\[ (u - l)^{[j]} = (u - l)(u - l - 1)\cdots(u - l - j + 1) \]

The polynomial has integer coefficients and highest term $u^{j}$.
Thus the integral between $0$ and $1$ is a sum of fractions with denominators $1, 2, \ldots (j+1)$.
So it can be written as a fraction with denominator $L(j) = gcd(1, 2, \ldots, (j+1))$.
Thus the denominator of $\beta_j$ divides $L(j)j!$, and Pickard defines an integer sequence
\[ \aleph_j = L(j)j!\beta_j \]

Thus the $\aleph_j$ can be worked out mechanically, either from the definite integral (\ref{eqn:beta_integral}), or using the Sterling numbers, which comes to the same thing, and the integer sequences $\aleph_j$ and $L(j)j!$ give the rational sequence $\beta_j$ as their ratio.



\bibliography{numerical}
\bibliographystyle{plain}

\end{document}

