\documentclass{article}

\usepackage{hyperref}
\usepackage{amsmath}
\usepackage{amsfonts}
\usepackage{amssymb}

\title{Coefficients for numerical integration as defined by Pickard}
\author{Jack Grahl}

\begin{document}
\maketitle
\abstract{The definition of some integer sequences and triangles in \cite{pickard}, and the relationship between these definitions, and some similar notions in other works on numerical integration, is a little unclear. We try to make the definitions precise and to provide a concordance with other notations.}

\section{Pickard's paper}
\subsection{$\aleph_j$ and $\aleph^{\star}_j$}
The article \cite{pickard} provided the sequences A002397 to A002406. Most of these sequences are taken from the tables for $\delta_p(J)$ and $\delta^{\star}_p(J)$. These are defined for all $0 \leq J$ and $0 \leq p \leq J$.

Pickard first defines (in (6a)):
\begin{equation}\label{eqn:beta_integral}
 \beta_j = \frac{1}{j!}\int_0^1 (u + j - 1)^{[j]} du 
\end{equation}
where the integrand is the polynomial defined by:
\[ (u - l)^{[j]} = (u - l)(u - l - 1)\cdots(u - l - j + 1) \]
The goal is to use $\beta_j$ in the estimate:
\[ y(x_1) = y(x_0) + h\sum_{j=0}^{J} \beta_j \Delta^{j}f_{-j} + E_{J} \]
where $E_J$ is an error term, and $\Delta$ is a forward difference operator.

The polynomial has integer coefficients and highest term $u^{j}$.
Thus the integral between $0$ and $1$ is a sum of fractions with denominators $1, 2, \ldots (j+1)$.
So it can be written as a fraction with denominator $L(j) = gcd(1, 2, \ldots, (j+1))$.
Thus the denominator of $\beta_j$ divides $L(j)j!$, and Pickard defines an integer sequence
\begin{equation}\label{eqn:aleph}
 \aleph_j = L(j)j!\beta_j 
\end{equation}

Thus the $\aleph_j$ can be worked out mechanically, either from the definite integral (\ref{eqn:beta_integral}), or using the Stirling numbers, which comes to the same thing, and the integer sequences $\aleph_j$ and $L(j)j!$ give the rational sequence $\beta_j$ as their ratio.

An entirely analogous set of definitions give $\beta^{\star}_j$ and $\aleph^{\star}_j$:
\[ \beta^{\star}_j = \frac{1}{j!}\int_{-1}^0 (u + j - 1)^{[j]} du \]
\[ \aleph^{\star}_j = L(j)j!\beta^{\star}_j \]
\[ y(x_0) = y(x_1) + h\sum_{j=0}^{J} \beta^{\star}_j \Delta^{j}f_{-j} + E^{\star}_{J} \]

Here the only difference between these equations and the corresponding ones is that the estimate now gives $y(x_0)$ in terms of $y(x_1)$ and not vice versa, and the change in the integration bounds from $(0, 1)$ to $(-1, 0)$.

\subsection{$\delta_p(J)$ and $\delta^{\star}_p(J)$}
In equation (12) of \cite{pickard}, the symbol $\gamma_{p,j}$ (written without a comma in the subscript by Pickard) is implicitly defined
\[ \Delta^{j}f_{-j}  = \sum_{p=0}^{j} \gamma_{p,j}f_{-p} \]
Thus powers of the difference operator are expanded in the tabular points of $f$.
The coefficients of this expansion are the alternating binomial coefficients:

\begin{center}
\begin{tabular}{ccccccccc}
&&&1&&-1&&&\\
&&1&&-2&&1&&\\
&1&&-3&&3&&-1&\\
1&&-4&&6&&-4&&1\\
\end{tabular}
\end{center}
or equivalently,
\[ \gamma_{p, j} = (-1)^{p} {j \choose p} \]

This allows terms of the form $\beta_j \Delta^{j}f_{-j}$ to be written $\sum_{p=0}^{j} \gamma_{p, j} \beta_{j} f_{-p}$.
We then define 
\[ \alpha_p(J) = \sum_{j=p}^{J} \gamma_{p, j} \beta{j} \]
which allows us to further simplify the estimate for $y(x_1)$ given in Equation (4) as
\[ y(x_1) = y(x_0) = h\sum_{p=0}^{J} \alpha_{p}(J)f_{-p} + E_{J} \]

Like $\beta{j}$, $\alpha{j}$ are fractions with a denominator dividing $L(j)j!$.
As before, it is convenient to multiply out by this denominator to get integers, here $\delta_p(J) = L(J)J! \alpha_p(J)$.

To relate $\delta_p(J)$ to $\aleph_j$, we have
\[ \delta_p(J) = L(J)J! \sum_{j=p}^{J} \gamma_{p, j} \beta_{j} \]
and therefore
\[ \delta_p(J) = \sum_{j=p}^{J} \frac{L(J)J!}{L(j)j!} \gamma_{p, j} \aleph_{j} \]
where all the terms on the right hand side, including the fraction, are integers.
This, together with the values of $\aleph_{j}$ given by (\ref{eqn:beta_integral} and \ref{eqn:aleph}, allow us to calculate arbitrary values of $\delta_p(J)$.

An exactly analogous consideration exists for $\delta^{\star}_p(J)$.

\section{Concordance}
Pickard cites both the books \cite{collatz} and \cite{henrici} as providing tables of the coefficients $\beta_j$ and $\beta^{\star}_j$, albeit with fewer entries than his own.

\section{Further notes}
The paper \cite{curtz}, which we have not been able to get hold of, certainly defines a similar collection of coefficients. The sequence \url{https://oeis.org/A140825} and the paper \cite{flajolet} do not give us all the information we need to be sure of the exact relationship to those of Pickard.


\bibliography{numerical}
\bibliographystyle{plain}

\end{document}

